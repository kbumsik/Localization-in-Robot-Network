\section{Introduction}
\label{sec:introduction}
\subsection{Background}
	There are many robots created to improve quality of our life, so the AI of robots can make a great contribution. What's more, we may want to easily find a specific room location even when we are not familiar with the place. We can navigate our outdoor position with GPS, but it is hard for us to know our indoor position with our smart devices. All of these problems lead to the topic on how to implement the localization of our robots.
\par
	In terms of localization, our method to find direction first and measure the distance along with the direction. There are several algorithms of direction of arrival, we choose to use MUSIC, stands for MUiltiple SIgnal Classification, one of
the high resolution subspace DOA algorithms, which gives the estimation of number of signals arrived, hence their direction of arrival. Compare with other algorithm, it is able to estimate frequencies with accuracy higher than one sample, because its estimation function can be evaluated for any frequency. This is a form of superresolution. MUSIC estimates the frequency content of a signal or autocorrelation matrix using an eigenspace method. This method assumes that a signal, $x(n)$, consists of $p$ complex exponentials in the presence of Gaussian white noise. Given an $M \times M$ autocorrelation matrix, $\mathbf{R}_x$, if the eigenvalues are sorted in decreasing order, the eigenvectors corresponding to the p largest eigenvalues (i.e. directions of largest variability) span the signal subspace. The remaining $M-p$ eigenvectors span the orthogonal space, where there is only noise. Note that for $M = p + 1$, MUSIC is identical to Pisarenko harmonic decomposition. The general idea is to use averaging to improve the performance of the Pisarenko estimator. The equation of MUSIC will be described later.
\par
	After the procedure of DOA, we will estimate distance based on path loss with noise of Rayleigh fading and Gaussian white noise. Path loss is the reduction in power density of an electromagnetic wave as it propagates through space. This term is commonly used in wireless communications and signal propagation. Although path loss may be due to many effects, such as free-space loss, refraction, diffraction, reflection, aperture-medium coupling loss, and absorption, we decide to use simple version of pass loss equation without considering so many effects, which is commonly used and called free space propagation. The simplified equation will be described later.
\par
	In terms of noise simulation, we decide to use the effect of Rayleigh fading,a statistical model for the effect of a propagation environment on a radio signal, such as that used by wireless devices. Rayleigh fading models assume that the magnitude of a signal that has passed through such a transmission medium (also called a communications channel) will vary randomly, or fade, according to a Rayleigh distribution — the radial component of the sum of two uncorrelated Gaussian random variables. Rayleigh fading is most applicable when there is no dominant propagation along a line of sight between the transmitter and receiver. In addition to Rayleigh fading noise, we also add Gaussian white noise, which is alo known as normal distribution white noise. It is very commonly used in signal processing because of the central limit theorem, it states that averages of random variables independently drawn from independent distributions converge in distribution to the normal, that is, become normally distributed when the number of random variables is sufficiently large.  A random variable with a Gaussian distribution is said to be normally distributed and is called a normal deviate if mean of the distribution is 0 and its standard deviation is 1.


\subsection{Related Work}


%\begin{figure}[h]
%\centering
%\includegraphics[width=0.8\linewidth]{img/img1}
%\caption{3D terrain surface}
%\label{fig4}
%\end{figure} 

%application \cite{Sichitiu, Musolesi2009}
%realistic \cite{Jardosh2003}
%general survey \cite{Camp2002,Bai2004}
%group \cite{Hong1999}
%probability \cite{Mohimani2009}